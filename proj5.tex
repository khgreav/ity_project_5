\documentclass[hyperref={unicode}]{beamer}
\usetheme{Warsaw}
\usepackage{lmodern}
\usepackage[czech]{babel}
\usepackage[utf8]{inputenc}
\usepackage{amsmath}
\usepackage{amssymb}
\usepackage{amsthm}
\usepackage{tikz}
\usetikzlibrary{shadows}
\newtheorem*{customdefinition}{Definice}
\newtheorem*{customsentence}{Věta}

\title{Typografie a publikování}
\subtitle{Projekt č.5 - Konečné automaty}
\author{Karel Hanák}
\institute{Vysoké učení technické v Brně \\ Fakulta informačních technologií}
\date{\today}

\begin{document}
\frame{\titlepage}

\begin{frame}
\frametitle{Obsah}
\tableofcontents
\end{frame}

\section{Konečné automaty}
\begin{frame}
\frametitle{Co je konečný automat?}

Konečný automat model jednoduchého stroje, který se používá na rozhodnutí problému akceptování či zamítnutí slova z jazyka. \\~\\

\pause

Existují 2 základní druhy konečných automatů:
\begin{itemize}
\item Deterministický konečný automat
\item Nedeterministický konečný automat
\end{itemize}

\pause

\begin{customsentence}[1]\label{sen:1}
Ke každému deterministickému konečnému automatu existuje ekvivalentní nedeterministický konečný automat.
\end{customsentence}

\end{frame}

\section{Deterministický konečný automat}

\begin{frame}
\frametitle{Deterministický konečný automat - DFA}

\begin{customdefinition}[1]\label{def:1}
{\normalfont DFA} je uspořádaná pětice $(Q,\Sigma,\delta,q0,F)$, kde:
\begin{itemize}
\item $Q$ je konečná neprázdná množina stavů
\item $\Sigma$ je konečná neprázdná abeceda
\item $\delta$ je přechodová funkce ve tvaru $Q \times \Sigma \rightarrow Q$
\item $q0$ je počáteční stav, pro který platí $q0 \in Q$
\item $F$ je množina akceptujících stavů, pro kterou platí $F \subseteq Q$
\end{itemize}
\end{customdefinition}

DFA má pro každou dvojici stav a symbol pouze jeden přechod.

\end{frame}

\section{Nedeterministický konečný automat}

\begin{frame}
\frametitle{Nedeterministický konečný automat - NFA}

\begin{customdefinition}[2]\label{def:2}
{\normalfont NFA} je uspořádaná pětice $(Q,\Sigma,\delta,q0,F)$, kde:
\begin{itemize}
\item $Q$ je konečná neprázdná množina stavů
\item $\Sigma$ je konečná neprázdná abeceda
\item $\delta$ je přechodová funkce ve tvaru $Q \times \Sigma \cup \{ \varepsilon \} \rightarrow 2^Q$
\item $q0$ je počáteční stav, pro který platí $q0 \in Q$
\item $F$ je množina akceptujících stavů, pro kterou platí $F \subseteq Q$
\end{itemize}
\end{customdefinition}

NFA může mít pro každou dvojici stav a symbol libovolný počet přechodů.

\end{frame}

\section{Ukázka výpočtu}

\begin{frame}
\frametitle{Ukázka výpočtu}

\begin{block}{Příklad 1}
Mějme definován jazyk $L = \{ a.b^*.c \}$ a vstupní slovo $w = abbc$. \\
Níže je znázorněn výpočet DFA nad vstupním slovem.
\end{block}

\begin{center}
\begin{tikzpicture}

\draw (0,0) circle [x radius=1, y radius=1] node [align=center] {q0};
\draw[red] (1,0) -- (2,0);
\draw[red] (2,0) -- (1.8,0.2);
\draw[red] (2,0) -- (1.8,-0.2);
\draw[red, thick] (1.5,0.2) node {a};
\draw[red] (3,0) circle [x radius=1, y radius=1] node [align=center] {q1};
\draw[rounded corners=8pt] (3,1) -- (3.3,1.5) -- (3,2) -- (2.7,1.5) -- (2.8,1.3);
\draw (3.5,1.5) node {b};
\draw (2.8,1.3) -- (2.6,1.5);
\draw (2.8,1.3) -- (3,1.5);
\draw (4,0) -- (5,0);
\draw (5,0) -- (4.8,0.2);
\draw (5,0) -- (4.8,-0.2);
\draw (4.5,0.2) node {c};
\draw (6,0) circle [x radius=1, y radius=1] node [align=center] {F};
\draw (6,0) circle [x radius=0.9, y radius=0.9];

\end{tikzpicture}
\end{center}

Nejprve se provede pro dvojici (q0,a) přechod do q1.

\end{frame}

\begin{frame}
\frametitle{Ukázka výpočtu 2}

\begin{block}{Příklad 1}
Mějme definován jazyk $L = \{ a.b^*.c \}$ a vstupní slovo $w = abbc$. \\
Níže je znázorněn výpočet DFA nad vstupním slovem.
\end{block}

\begin{center}
\begin{tikzpicture}

\draw (0,0) circle [x radius=1, y radius=1] node [align=center] {q0};
\draw (1,0) -- (2,0);
\draw (2,0) -- (1.8,0.2);
\draw (2,0) -- (1.8,-0.2);
\draw (1.5,0.2) node {a};
\draw[red] (3,0) circle [x radius=1, y radius=1] node [align=center] {q1};
\draw[red, rounded corners=8pt] (3,1) -- (3.3,1.5) -- (3,2) -- (2.7,1.5) -- (2.8,1.3);
\draw[red, thick] (3.5,1.5) node {b};
\draw[red] (2.8,1.3) -- (2.6,1.5);
\draw[red] (2.8,1.3) -- (3,1.5);
\draw (4,0) -- (5,0);
\draw (5,0) -- (4.8,0.2);
\draw (5,0) -- (4.8,-0.2);
\draw (4.5,0.2) node {c};
\draw (6,0) circle [x radius=1, y radius=1] node [align=center] {F};
\draw (6,0) circle [x radius=0.9, y radius=0.9];

\end{tikzpicture}
\end{center}

Dále se pro dvojici (q1,b) provede přechod do q1 dvakrát.

\end{frame}

\begin{frame}
\frametitle{Ukázka výpočtu 3}

\begin{block}{Příklad 1}
Mějme definován jazyk $L = \{ a.b^*.c \}$ a vstupní slovo $w = abbc$. \\
Níže je znázorněn výpočet DFA nad vstupním slovem.
\end{block}

\begin{center}
\begin{tikzpicture}

\draw (0,0) circle [x radius=1, y radius=1] node [align=center] {q0};
\draw (1,0) -- (2,0);
\draw (2,0) -- (1.8,0.2);
\draw (2,0) -- (1.8,-0.2);
\draw (1.5,0.2) node {a};
\draw (3,0) circle [x radius=1, y radius=1] node [align=center] {q1};
\draw[rounded corners=8pt] (3,1) -- (3.3,1.5) -- (3,2) -- (2.7,1.5) -- (2.8,1.3);
\draw (3.5,1.5) node {b};
\draw (2.8,1.3) -- (2.6,1.5);
\draw (2.8,1.3) -- (3,1.5);
\draw[red] (4,0) -- (5,0);
\draw[red] (5,0) -- (4.8,0.2);
\draw[red] (5,0) -- (4.8,-0.2);
\draw[red] (4.5,0.2) node {c};
\draw[green] (6,0) circle [x radius=1, y radius=1] node [align=center] {F};
\draw[green] (6,0) circle [x radius=0.9, y radius=0.9];

\end{tikzpicture}
\end{center}

Nakonec se pro dvojici (q1,c) provede přechod do akceptujícího stavu F a výpočet je u konce.

\end{frame}

\section{Použité zdroje}

\begin{frame}
\frametitle{Použité zdroje}

\begin{itemize}
\item Konečný automat \\ {\footnotesize \url{https://matematika.cz/konecny-automat}}
\item Nedeterministický konečný automat \\ {\footnotesize \url{https://matematika.cz/nedeterministicky-konecny-automat}}
\end{itemize}
\end{frame}

\end{document}
